\documentclass[a4paper, 11pt, nofonts, nocap, fancyhdr]{ctexart}

\usepackage[margin=60pt]{geometry}

%\usepackage{fontspec}
%\setmainfont{Arial}
%\setsansfont{Arial}
%\setmonofont{Consolas}

\usepackage{xeCJK}
%\setCJKmainfont[BoldFont={FZDaHei-B02}, ItalicFont={FZXingKai-S04}]{FZLanTingHei-R-GBK}
%\setCJKsansfont{FZLanTingHei-R-GBK}
%\setCJKmonofont{FZLanTingHei-R-GBK}
\setCJKmainfont[BoldFont=STHeiti, ItalicFont=STKaiti]{STSong}
\setCJKsansfont[BoldFont=STHeiti]{STXihei}
\setCJKmonofont{STFangsong}

\usepackage{enumitem}
\setenumerate[1]{itemsep=0pt,partopsep=0pt,parsep=\parskip,topsep=0pt}
\setitemize[1]{itemsep=0pt,partopsep=0pt,parsep=\parskip,topsep=0pt}
\setdescription{itemsep=0pt,partopsep=0pt,parsep=\parskip,topsep=0pt}

\usepackage{listings}
\lstset{
	language=SQL,
	basicstyle=\ttfamily\small,
	breaklines=true,
	tabsize=4,
	showstringspaces=true,
	extendedchars=false
}

\usepackage{graphicx}
\usepackage{subfigure}
	\renewcommand\figurename{图}

\usepackage{ulem}
\usepackage[bookmarks=true,colorlinks,linkcolor=black]{hyperref}

\CTEXoptions[today=small]
\setcounter{secnumdepth}{4}

\pagestyle{plain}

\title{数据库引论 Project——数据库查询优化}
\author
{
	胡旻\\
	13307130086
	\and
	周吉\\
	13307130227
}
\date{\today}

\begin{document}

\maketitle
\tableofcontents
\setcounter{page}{0}
\thispagestyle{empty}
\newpage

\section{实验概述}

\subsection{主要内容}

给定大众点评中的店铺信息,将其分解至满足第三范式。然后在数据集上,针对不同类型的查询语句,进行分析优化。

\subsection{分析方式}

由于给定的信息数据实在太小,慢查询等优化手段无法体现效果,所以在实验中,主要采用了explain对查询语句进行分析,根据分析结果中的row,对查询的效率进行评判,从而找出一种优化的方式,并以此判断优化的比例。

\subsection{实验环境}

mysql  Ver 14.14 Distrib 5.6.24

Mac OS

\section{实验内容}

\subsection{Task 1}

对数据进行分析以后,可以分成4个表:

\begin{itemize}
    \item CITY : cityi, city, province
    \item ADDRESS : originallatitude, originallongitude, address, area, businessarea, cityi
    \item Tag : tags, shopid
    \item SHOP : shopid, name, alias, cityi, phone, hours, avgprice, stars, photos, description, tags, navigation, characteristics, productrating, environmentrating, servicerating, verygoodremarks, goodremarks, commonremarks, badremarks, verybadremarks, recommendeddishes, ischains, groupon, card
\end{itemize}

根据第三范式的规则,以上拆分满足第三范式。

\subsection{Task 2}

\subsubsection{单表查询}

\textbf{(1) 查询表中的所有字段} 
      
explain select * from city;

查询全部city中的字段,所以至少读取59行。(0.00123325s)

\begin{verbatim}
+----+-------------+-------+------+---------------+------+---------+------+------+-------+
| id | select_type | table | type | possible_keys | key  | key_len | ref  | rows | Extra |
+----+-------------+-------+------+---------------+------+---------+------+------+-------+
|  1 | SIMPLE      | city  | ALL  | NULL          | NULL | NULL    | NULL |   59 | NULL  |
+----+-------------+-------+------+---------------+------+---------+------+------+-------+
\end{verbatim}

无法优化

\vspace{0.7cm}

\textbf{(2) 查询表中的指定字段} 

explain select name from city;(0.00054675s)

查询全部city中的name字段,所以至少读取59行。 

\begin{verbatim}
+----+-------------+-------+------+---------------+------+---------+------+------+-------+
| id | select_type | table | type | possible_keys | key  | key_len | ref  | rows | Extra |
+----+-------------+-------+------+---------------+------+---------+------+------+-------+
|  1 | SIMPLE      | city  | ALL  | NULL          | NULL | NULL    | NULL |   59 | NULL  |
+----+-------------+-------+------+---------------+------+---------+------+------+-------+
\end{verbatim}

无法优化

\vspace{0.7cm}

\textbf{(3) 查询表中没有重复的字段(distinct)的使用} 

explain select distinct province from city;

查询不重名的所有省份(0.04483825s)

\begin{verbatim}
+----+-------------+-------+------+---------------+------+---------+------+------+-------+
| id | select_type | table | type | possible_keys | key  | key_len | ref  | rows | Extra |
+----+-------------+-------+------+---------------+------+---------+------+------+-------+
|  1 | SIMPLE      | city  | ALL  | NULL          | NULL | NULL    | NULL |   59 | NULL  |
+----+-------------+-------+------+---------------+------+---------+------+------+-------+
\end{verbatim}

无法优化

\vspace{0.7cm}

\textbf{(4) 条件查询各表主键的字段(单值查询或范围查询)} 

i.explain select * from city where cid=1;

查询cid为1的城市

\begin{verbatim}
+----+-------------+-------+-------+---------------+---------+---------+-------+------+-------+
| id | select_type | table | type  | possible_keys | key     | key_len | ref   | rows | Extra |
+----+-------------+-------+-------+---------------+---------+---------+-------+------+-------+
|  1 | SIMPLE      | city  | const | PRIMARY       | PRIMARY | 4       | const |    1 | NULL  |
+----+-------------+-------+-------+---------------+---------+---------+-------+------+-------+
\end{verbatim}

主键本身就是一个索引,所以无法优化

ii.explain select * from city where cid>1 and cid<10;

查询 cid 在1到10之间的城市 

\begin{verbatim}
+----+-------------+-------+-------+---------------+---------+---------+------+------+-------------+
| id | select_type | table | type  | possible_keys | key     | key_len | ref  | rows | Extra       |
+----+-------------+-------+-------+---------------+---------+---------+------+------+-------------+
|  1 | SIMPLE      | city  | range | PRIMARY       | PRIMARY | 4       | NULL |    7 | Using where |
+----+-------------+-------+-------+---------------+---------+---------+------+------+-------------+
\end{verbatim}

无需优化

\vspace{0.7cm}

\textbf{(5) 条件查询各表中普通字段(单值查询或范围查询)} 

explain select * from city where province='浙江';

查询浙江省的全部信息,扫描了59行。

\begin{verbatim}
+----+-------------+-------+------+---------------+------+---------+------+------+-------------+
| id | select_type | table | type | possible_keys | key  | key_len | ref  | rows | Extra       |
+----+-------------+-------+------+---------------+------+---------+------+------+-------------+
|  1 | SIMPLE      | city  | ALL  | NULL          | NULL | NULL    | NULL |   59 | Using where |
+----+-------------+-------+------+---------------+------+---------+------+------+-------------+
\end{verbatim}

create index index\_province on city(province);

explain select * from city where province='浙江';

\begin{verbatim}
+----+-------------+-------+------+----------------+----------------+---------+-------+------+-----------------------+
| id | select_type | table | type | possible_keys  | key            | key_len | ref   | rows | Extra                 |
+----+-------------+-------+------+----------------+----------------+---------+-------+------+-----------------------+
|  1 | SIMPLE      | city  | ref  | index_province | index_province | 33      | const |    6 | Using index condition |
+----+-------------+-------+------+----------------+----------------+---------+-------+------+-----------------------+
\end{verbatim}

建立以province为关键字的索引,可以看到估计所用的行数大大减少了,只有原来的1/10。

\vspace{0.7cm}

\textbf{(6) 一个表中多个字段条件查询(单值查询或范围查询)} 

explain select sid from shop where good\_remarks>20 and very\_bad\_remarks<10;

查询shop id,好评大于20 极差小于10 的情况。读取行数979行。

\begin{verbatim}
+----+-------------+-------+------+---------------+------+---------+------+------+-------------+
| id | select_type | table | type | possible_keys | key  | key_len | ref  | rows | Extra       |
+----+-------------+-------+------+---------------+------+---------+------+------+-------------+
|  1 | SIMPLE      | shop  | ALL  | NULL          | NULL | NULL    | NULL |  979 | Using where |
+----+-------------+-------+------+---------------+------+---------+------+------+-------------+
\end{verbatim}

优化:

create index index\_goodremarks on shop(good\_remarks);

create index index\_verybadremarks on shop(very\_bad\_remarks);

explain select sid from shop where good\_remarks>20 and very\_bad\_remarks<10;

\begin{verbatim}
+----+-------------+-------+-------+----------------------------------------+-------------------+---------+------+------+------------------------------------+
| id | select_type | table | type  | possible_keys                          | key               | key_len | ref  | rows | Extra                              |
+----+-------------+-------+-------+----------------------------------------+-------------------+---------+------+------+------------------------------------+
|  1 | SIMPLE      | shop  | range | index_goodremarks,index_verybadremarks | index_goodremarks | 5       | NULL |  178 | Using index condition; Using where |
+----+-------------+-------+-------+----------------------------------------+-------------------+---------+------+------+------------------------------------+
\end{verbatim}

建立了两个索引,大大减少了查询的行数,变成了1/5。

\vspace{0.7cm}

\textbf{(7) 用”in”进行条件查询} 

i. explain select province from city where cid in ('1', '2', '3');

查询city id 为1,2,3的省份,因为是主键就有索引。所以只有3行查询。

\begin{verbatim}
+----+-------------+-------+-------+---------------+---------+---------+------+------+-------------+
| id | select_type | table | type  | possible_keys | key     | key_len | ref  | rows | Extra       |
+----+-------------+-------+-------+---------------+---------+---------+------+------+-------------+
|  1 | SIMPLE      | city  | range | PRIMARY       | PRIMARY | 4       | NULL |    3 | Using where |
+----+-------------+-------+-------+---------------+---------+---------+------+------+-------------+
\end{verbatim}

无需优化

ii. explain select name from city where province in ('浙江', '广州');

查询省份是浙江,广州的城市。读取了59行(全部数据)。

\begin{verbatim}
+----+-------------+-------+------+---------------+------+---------+------+------+-------------+
| id | select_type | table | type | possible_keys | key  | key_len | ref  | rows | Extra       |
+----+-------------+-------+------+---------------+------+---------+------+------+-------------+
|  1 | SIMPLE      | city  | ALL  | NULL          | NULL | NULL    | NULL |   59 | Using where |
+----+-------------+-------+------+---------------+------+---------+------+------+-------------+
\end{verbatim}

create index index\_province on city(province);

explain select name from city where province in ('浙江', '广州');

\begin{verbatim}
+----+-------------+-------+-------+----------------+----------------+---------+------+------+-----------------------+
| id | select_type | table | type  | possible_keys  | key            | key_len | ref  | rows | Extra                 |
+----+-------------+-------+-------+----------------+----------------+---------+------+------+-----------------------+
|  1 | SIMPLE      | city  | range | index_province | index_province | 33      | NULL |    7 | Using index condition |
+----+-------------+-------+-------+----------------+----------------+---------+------+------+-----------------------+
\end{verbatim}

优化:以province建立索引以后,估计读取行数减少为1/8。

\vspace{0.7cm}

\textbf{(8) 一个表中 group by、order by、having 联合条件查询} 

explain select * from shop where avg\_price<20 order by avg\_price;

排序avg\_price,找到小于20的商铺信息。读取了全部数据。

\begin{verbatim}
+----+-------------+-------+------+---------------+------+---------+------+------+-----------------------------+
| id | select_type | table | type | possible_keys | key  | key_len | ref  | rows | Extra                       |
+----+-------------+-------+------+---------------+------+---------+------+------+-----------------------------+
|  1 | SIMPLE      | shop  | ALL  | NULL          | NULL | NULL    | NULL |  979 | Using where; Using filesort |
+----+-------------+-------+------+---------------+------+---------+------+------+-----------------------------+
\end{verbatim}

create index price on shop(avg\_price);

explain select * from shop where avg\_price<20 order by avg\_price;

\begin{verbatim}
+----+-------------+-------+-------+---------------+-------+---------+------+------+-----------------------+
| id | select_type | table | type  | possible_keys | key   | key_len | ref  | rows | Extra                 |
+----+-------------+-------+-------+---------------+-------+---------+------+------+-----------------------+
|  1 | SIMPLE      | shop  | range | price         | price | 5       | NULL |  233 | Using index condition |
+----+-------------+-------+-------+---------------+-------+---------+------+------+-----------------------+
\end{verbatim}

优化:建立avg\_price索引,读取次数减少到了1/4.

\subsubsection{复合查询}

\textbf{(1) 多表联合查询} 

explain select sid from shop, city, address where shop.address=address.address and address.cid=city.cid and city.cid=1;

合并shop、city、address两个表,查找出cid = 1的商铺编号。

\begin{verbatim}
+----+-------------+---------+-------+---------------+---------+---------+-------+------+----------------------------------------------------+
| id | select_type | table   | type  | possible_keys | key     | key_len | ref   | rows | Extra                                              |
+----+-------------+---------+-------+---------------+---------+---------+-------+------+----------------------------------------------------+
|  1 | SIMPLE      | city    | const | PRIMARY       | PRIMARY | 4       | const |    1 | Using index                                        |
|  1 | SIMPLE      | address | ref   | cid           | cid     | 5       | const |  255 | NULL                                               |
|  1 | SIMPLE      | shop    | ALL   | NULL          | NULL    | NULL    | NULL  |  961 | Using where; Using join buffer (Block Nested Loop) |
+----+-------------+---------+-------+---------------+---------+---------+-------+------+----------------------------------------------------+
\end{verbatim}

create index index\_address on address(address);

explain select sid from shop, city, address where shop.address=address.address and address.cid=city.cid and city.cid=1;

\begin{verbatim}
+----+-------------+---------+-------+-------------------+---------------+---------+---------------------+------+-------------+
| id | select_type | table   | type  | possible_keys     | key           | key_len | ref                 | rows | Extra       |
+----+-------------+---------+-------+-------------------+---------------+---------+---------------------+------+-------------+
|  1 | SIMPLE      | city    | const | PRIMARY           | PRIMARY       | 4       | const               |    1 | Using index |
|  1 | SIMPLE      | shop    | ALL   | NULL              | NULL          | NULL    | NULL                |  961 | Using where |
|  1 | SIMPLE      | address | ref   | cid,index_address | index_address | 303     | sqlopt.shop.address |    1 | Using where |
+----+-------------+---------+-------+-------------------+---------------+---------+---------------------+------+-------------+
\end{verbatim}

优化建立address的索引,因为cid是主键已经有索引,所以address和city只需查找一行。

\vspace{0.7cm}

\textbf{(2) join 查询} 

\vspace{0.7cm}

\textbf{(3) 存在量词(exists)查询} 

explain select cid from address where exists (select * from shop, address where shop.address=address.address and shop.avg\_price<20);

\begin{verbatim}
+----+-------------+---------+-------+---------------+---------------+---------+---------------------+------+------------------------------------+
| id | select_type | table   | type  | possible_keys | key           | key_len | ref                 | rows | Extra                              |
+----+-------------+---------+-------+---------------+---------------+---------+---------------------+------+------------------------------------+
|  1 | PRIMARY     | address | index | NULL          | cid           | 5       | NULL                |  996 | Using index                        |
|  2 | SUBQUERY    | shop    | range | price         | price         | 5       | NULL                |  233 | Using index condition; Using where |
|  2 | SUBQUERY    | address | ref   | index_address | index_address | 303     | sqlopt.shop.address |    1 | Using index                        |
+----+-------------+---------+-------+---------------+---------------+---------+---------------------+------+------------------------------------+
\end{verbatim}

\vspace{0.7cm}

\textbf{(4) 嵌套子查询(select … from (select …))} 

explain select name from (select * from shop where avg\_price < 20) as D;

\begin{verbatim}
+----+-------------+------------+------+---------------+------+---------+------+------+-------------+
| id | select_type | table      | type | possible_keys | key  | key_len | ref  | rows | Extra       |
+----+-------------+------------+------+---------------+------+---------+------+------+-------------+
|  1 | PRIMARY     | <derived2> | ALL  | NULL          | NULL | NULL    | NULL |  961 | NULL        |
|  2 | DERIVED     | shop       | ALL  | NULL          | NULL | NULL    | NULL |  961 | Using where |
+----+-------------+------------+------+---------------+------+---------+------+------+-------------+
\end{verbatim}

create index price on shop(avg\_price);

explain select name from (select * from shop where avg\_price < 20) as D;

\begin{verbatim}
+----+-------------+------------+-------+---------------+-------+---------+------+------+-----------------------+
| id | select_type | table      | type  | possible_keys | key   | key_len | ref  | rows | Extra                 |
+----+-------------+------------+-------+---------------+-------+---------+------+------+-----------------------+
|  1 | PRIMARY     | <derived2> | ALL   | NULL          | NULL  | NULL    | NULL |  233 | NULL                  |
|  2 | DERIVED     | shop       | range | price         | price | 5       | NULL |  233 | Using index condition |
+----+-------------+------------+-------+---------------+-------+---------+------+------+-----------------------+
\end{verbatim}

change to:

explain select name from shop where avg\_price < 20;

\begin{verbatim}
+----+-------------+-------+-------+---------------+-------+---------+------+------+-----------------------+
| id | select_type | table | type  | possible_keys | key   | key_len | ref  | rows | Extra                 |
+----+-------------+-------+-------+---------------+-------+---------+------+------+-----------------------+
|  1 | SIMPLE      | shop  | range | price         | price | 5       | NULL |  233 | Using index condition |
+----+-------------+-------+-------+---------------+-------+---------+------+------+-----------------------+
\end{verbatim}

\subsubsection{其他查询}

\textbf{(1) 向表中插入记录} 

explain insert into city values('test', 'test', 11111);

\begin{verbatim}
+----+-------------+-------+------+---------------+------+---------+------+------+----------------+
| id | select_type | table | type | possible_keys | key  | key_len | ref  | rows | Extra          |
+----+-------------+-------+------+---------------+------+---------+------+------+----------------+
|  1 | SIMPLE      | NULL  | NULL | NULL          | NULL | NULL    | NULL | NULL | No tables used |
+----+-------------+-------+------+---------------+------+---------+------+------+----------------+
\end{verbatim}

\vspace{0.7cm}

\textbf{(2) 向表中删除记录} 

explain delete from city where name='test';

\begin{verbatim}
+----+-------------+-------+------+---------------+------+---------+------+------+-------------+
| id | select_type | table | type | possible_keys | key  | key_len | ref  | rows | Extra       |
+----+-------------+-------+------+---------------+------+---------+------+------+-------------+
|  1 | SIMPLE      | city  | ALL  | NULL          | NULL | NULL    | NULL |   59 | Using where |
+----+-------------+-------+------+---------------+------+---------+------+------+-------------+
\end{verbatim}

create index index\_name on city(name);

explain delete from city where name='test';

\begin{verbatim}
+----+-------------+-------+-------+---------------+------------+---------+-------+------+-------------+
| id | select_type | table | type  | possible_keys | key        | key_len | ref   | rows | Extra       |
+----+-------------+-------+-------+---------------+------------+---------+-------+------+-------------+
|  1 | SIMPLE      | city  | range | index_name    | index_name | 303     | const |    1 | Using where |
+----+-------------+-------+-------+---------------+------------+---------+-------+------+-------------+
\end{verbatim}

\subsection{Task 3}

现在的数据库设计中,我们大量使用了varchar类型,我们可以将其改为char,在速度上会有一定的提升,但是由于数据规模实在太小,所以并没有任何明显的变化。另外,还可以对表中的一些数字变量添加索引,这样可以加快查找的速度。

\section{实验总结}

\end{document}
